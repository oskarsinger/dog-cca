\input{../../hosinger/tex/oskar_macros.tex}

\documentclass{article}

\usepackage[margin=1in]{geometry}
\usepackage{times}
\usepackage{amsmath}
\usepackage{amsthm}
\usepackage{amssymb}
\usepackage{natbib}
\usepackage{bbm}
\usepackage{algorithm}
\usepackage{algorithmic}

\newtheorem{thm}{Theorem}

\begin{document}

\section{Intro} \label{sec:intro}

\subsection{Notation} \label{subsec:notation}

In the following notes, $\mbX_i \in \mbbR^{n \times p_i}$ will refer to the observation matrix for the $i$-th 'view' of some assumed underlying phenomenon, where $n$ is the number of observations and $p_i$ is the ambient dimension of the $i$-th view. $\mbPhi_i \in \mbbR^{p_i \times k}$ will refer to some linear transform of $\mbX_i$ into a $k$-dimensional vector space. The $k \times k$ identity matrix is denoted $\mbI_k$.

\subsection{Bregman CCA} \label{subsec:bregmancca}

Basic idea is replace the Frob norm in the CCA objective with a Bregman divergence.
	
	\bibliographystyle{apalike}
	\bibliography{../../hosinger/tex/citations.bib}
\end{document}
